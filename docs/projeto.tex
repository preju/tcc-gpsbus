\documentclass[dvips,12pt]{article}

% Any percent sign marks a comment to the end of the line

% Every latex document starts with a documentclass declaration like this
% The option dvips allows for graphics, 12pt is the font size, and article
%   is the style

\usepackage[pdftex]{graphicx}
\usepackage{url}

% These are additional packages for "pdflatex", graphics, and to include
% hyperlinks inside a document.

\setlength{\oddsidemargin}{0.25in}
\setlength{\textwidth}{6.5in}
\setlength{\topmargin}{0in}
\setlength{\textheight}{8.5in}

% These force using more of the margins that is the default style

\begin{document}

% Everything after this becomes content
% Replace the text between curly brackets with your own

\title{Implementação de um aplicativo para localização dos ônibus em tempo real}
\author{Your group members names here\\
\textbf{Orientador:} Fernando Henrique e Paula da Luz
}
\date{\today}

% You can leave out "date" and it will be added automatically for today
% You can change the "\today" date to any text you like


\maketitle

% This command causes the title to be created in the document

\section{Introdução}

% An article style is separated into sections and subsections with 
%   markup such as this.  Use \section*{Principles} for unnumbered sections.
A idealização do projeto é devido ao trabalho de conclusão de curso, onde o grupo terá de fazer uma apresentação com enfoque em tecnologia e desenvolvimento de softwares.
Buscando inovar, onde pudesse implementar as ferramentas condizentes com a área do curso, foi elaborada a ideia de realizar algo que não há no Brasil. O projeto teve como base pesquisas feitas na internet e no mercado, visando que o projeto pode ir além de um trabalho acadêmico, sendo possível até gerar mais lucro, com usuários usando com mais frequência o transporte público, pois usando o aplicativo teriam a ideia onde o seu transporte passaria, quantas linhas estão circulando no momento e onde estão localizadas, gerando despreocupação por utilização de taxis e outros meios de transporte mais caros.
A ideia em si, seria o grupo desenvolver, através de um ``GPS'', um sistema em tempo real de navegação por satélite, com um smartphone que recebera da central as informações sobre a posição das rotas de ônibus da empresa ®Viação Piracicabana, onde as pessoas terão mais comodidade referente as rotas e horários dos ônibus. 
Visando aplicativos de sucesso, como um exemplo o Bike Santos, Moovit e Onde está meu ônibus?, porem a diferença, levando em conta os dois últimos, o sistema não será alimentado através de crowdsourcing (alimentado com informações do passageiro em tempo real), e sim por meio do próprio banco de dados da empresa, assim como é feito no site da SPTrans, mas com um sistema via mobile (a SPTrans não disponibiliza um meio amigável, simples e prático, sendo possível a visualização somente pelo site próprio da companhia). O grupo acredita que o retorno  na baixada santista será positivo, rodando em todas as plataformas mobile, gerando o uso constante do aplicativo. 

\end{document}
