\chapter{Introducao} % (fold)
\label{cha:introducao}

Falar sobre o aumento da frota de veiculos (achar um artigo que comente sobre isso.)

Comentar estudos sobre a mobilidade urbana (referenciar com artigos)

Falar sobre a necessidade de uso do transporte coletivo, e de que forma eles podem auxiliar no processo de mobilidade urbana.

Buscar alternativas que deram certo em outras cidades, usando novamente referencias em artigos.

Atualmente a sociedade sofre cada vez mais com o aumento excessivo de ve�culos, principalmente nas grandes cidades. Devido a isso, v�rias partes do pa�s v�m tentando encontrar uma solu��o cab�vel para este problema, atrav�s de ciclovias, cal�adas confort�veis, niveladas, assim pensando tamb�m nos cadeirantes, rod�zios municipais, fazendo com que o cidad�o deixe o carro na garagem e tenha a certeza de que chegar� ao local e no hor�rio desejado, meios de transporte p�blico mais acess�veis, utilizando m�todos mais acess�veis e informatizados como por exemplo aplicativo de localiza��o para metr�s e �nibus. Podemos chamar isto de mobilidade urbana. O conceito de mobilidade urbana � a necessidade de se locomover de um local ao outro, independentemente do ve�culo utilizado, com mais facilidade n�o enfrentando futuros problemas. 
% chapter introducao (end){Introdu��o}

