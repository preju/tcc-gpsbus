\chapter{Introducao} % (fold)
\label{cha:introducao}

A evolu��o da sociedade vem promovendo gradativamente diversas mudan�as entre elas est�o o aumento excessivo da popula��o, junto a falta de infraestrutura (ruas esburacadas, falta de sinaliza��o), m� localiza��o de moradia e trabalho, onde resultam em causas que afetam trafego urbano diariamente e provocam falta de organiza��o nas metr�poles, a soma desses fatores implicam na mobilidade urbana. 
No Brasil, a mobilidade urbana � uma �rea com infraestrutura prec�ria, onde o cidad�o, na grande maioria tem que se mover de um ponto da cidade para outro, necessitando de qualidade e rapidez, todavia essa  infraestrutura n�o segue o crescimento conforme a demanda, gerando transtornos como tr�nsito excessivo, congestionamento e acidentes. Como de costume parte da popula��o usufrui do transporte coletivo, devido as condi��es financeiras e facilidade de acesso, h� de se ressaltar tamb�m que esse tipo de transporte, acarreta em diversos problemas como superlota��o, falta de qualidade e seguran�a.
O avan�o computacional vem contribuindo para minimizar alguns destes problemas citados acima. Com a tecnologia � poss�vel realizar um melhor planejamento urbano, realizando buscas atrav�s de mapas ou tecnologia GPS, pode-se obter localiza��o exata de vias, pontos de congestionamentos ou acidentes, localiza��o de transportes p�blicos, para melhor escolher seu trajeto e melhorar consideravelmente o modo em que a informa��o circula.

% chapter introducao (end){Introdu��o}

