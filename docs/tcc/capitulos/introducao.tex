\chapter{Introdução} % (fold)
\label{cha:introducao}

A evolução da sociedade vem promovendo gradativamente diversas mudanças entre elas estão o aumento excessivo da população, junto a falta de infraestrutura (ruas esburacadas, falta de sinalização), má localização de moradia e trabalho, onde resultam em causas que afetam trafego urbano diariamente e provocam falta de organização nas metrópoles, a soma desses fatores implicam na mobilidade urbana.

No Brasil, a mobilidade urbana é uma área com infraestrutura precária, onde o cidadão, na grande maioria tem que se mover de um ponto da cidade para outro, necessitando de qualidade e rapidez, todavia essa  infraestrutura não segue o crescimento conforme a demanda, gerando transtornos como trânsito excessivo, congestionamento e acidentes. Como de costume parte da população usufrui do transporte coletivo, devido as condições financeiras e facilidade de acesso, há de se ressaltar também que esse tipo de transporte, acarreta em diversos problemas como superlotação, falta de qualidade e segurança.

O avanço computacional vem contribuindo para minimizar alguns destes problemas citados acima. Com a tecnologia é possível realizar um melhor planejamento urbano, realizando buscas através de mapas ou tecnologia GPS, pode-se obter localização exata de vias, pontos de congestionamentos ou acidentes, localização de transportes públicos, para melhor escolher seu trajeto e melhorar consideravelmente o modo em que a informação circula.

% chapter introducao (end){Introdução}

