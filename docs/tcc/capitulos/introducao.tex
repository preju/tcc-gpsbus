\documentclass{article}

\title{Introducao}
\author{Fernando Luz, Rafael Magalhães, Iaco Cesar, Thiago Morano}
\date{Fevereiro 2015}
\begin{document}
\maketitle
\section{Introducao}
Atualmente a sociedade sofre cada vez mais com o aumento excessivo de veículos, principalmente nas grandes cidades. Devido a isso, várias partes do país vêm tentando encontrar uma solução cabível para este problema, através de ciclovias, calçadas confortáveis, niveladas, assim pensando também nos cadeirantes, rodízios municipais, fazendo com que o cidadão deixe o carro na garagem e tenha a certeza de que chegará ao local e no horário desejado, meios de transporte público mais acessíveis, utilizando métodos mais acessíveis e informatizados como por exemplo aplicativo de localização para metrôs e ônibus. Podemos chamar isto de mobilidade urbana. O conceito de mobilidade urbana é a necessidade de se locomover de um local ao outro, independentemente do veículo utilizado, com mais facilidade não enfrentando futuros problemas. 
\end{document}
% subsection de_1800_a_1900 (end)

% section historia (end)
% chapter introducao (end)