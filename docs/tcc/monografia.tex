\documentclass[a4paper,capchap,espacoduplo,normaltoc]{abntunip}

%\usepackage[bookmarks,pdftex,a4paper,colorlinks=true,citecolor=black,urlcolor=blue,linkcolor=black,pdfpagemode=None]{hyperref}
\usepackage[bookmarks,colorlinks=true,citecolor=black,urlcolor=blue,linkcolor=black,pdfpagemode=UseNone]{hyperref}
\usepackage[centertags]{amsmath}
\usepackage{amsfonts}
\usepackage{amssymb}
\usepackage{amsthm}
\usepackage[T1]{fontenc}
\usepackage[latin1]{inputenc}
\usepackage[brazil]{babel}
\usepackage[alf,abnt-repeated-author-omit=yes]{abntex2cite}
\usepackage{url}
%\usepackage{winfonts}
\usepackage{txfonts}

\fontfamily{arial}\selectfont
\renewcommand{\rmdefault}{arial}

% Math -------------------------------------------------------------------
\newtheorem{theorem}{Teorema}{\bfseries}{\itshape}
\newtheorem{lemma}{Lema}{\bfseries}{\itshape}
\newtheorem{definition}{Defini��o}{\bfseries}{\itshape}
\newtheorem{corollary}{Corol�rio}{\bfseries}{\itshape}
\newtheoremstyle{example}{\topsep}{\topsep}%
	{}%         Body font
	{}%         Indent amount (empty = no indent, \parindent = para indent)
	{\bfseries}% Thm head font
	{:}%        Punctuation after thm head
	{.5em}%     Space after thm head (\newline = linebreak)
	{\thmname{#1}\thmnumber{ #2}\thmnote{ #3}}%         Thm head spec
\theoremstyle{example}
\newtheorem{example}{Exemplo}

\sloppy

\begin{document}

%           #1          #2             #3           #4              #5
%\autorPoli{firstnames}{firstinitials}{middlenames}{middleinitials}{surname}
\autorPoli{Nome}{N.}{Meio}{M.}{Sobrenome}

\titulo{GPS e �nibus}

\orientador{Fernando Henrique e Paula da Luz}

\monografiaFormatura
%\monografiaMBA
%\qualificacaoMSc{<�rea do Mestrado>}
%\qualificacaoMSc{Enge\-nharia El�trica}
%\dissertacao{<�rea do Mestrado>}
%\qualificacaoDr{<�rea do Mestrado>}
%\teseDr{<�rea do Doutorado>}
%\teseLD
%\memorialLD

%\areaConcentracao{<�rea de Concentra��o>}
\areaConcentracao{Sistemas Digitais}

%\departamento{<Departamento>}
\departamento{Instituto de Ci�ncias Exatas e Tecnologia}

\local{Santos}

\data{2015}

\dedicatoria{}

\capa{}

\folhaderosto{}

% Ficha Catalogr�fica

%\setboolean{PoliRevisao}{true} % gera o quadro de revis�o ap�s a defesa
\renewcommand{\PoliFichaCatalograficaData}{%
  1. Assunto \#1. 2. Assunto \#2. 3. Assunto \#3.
  I. Universidade de S�o Paulo. Escola Polit�cnica.
  \PoliDepartamentoData. II. t.}

\fichacatalografica % formata a ficha

\paginadedicatoria{}

\begin{agradecimentos}
\end{agradecimentos}

\begin{resumo}
\end{resumo}

\begin{abstract}
\end{abstract}

% \begin{resume}
% \end{resume}
% 
% \begin{zusammenfassung}
% \end{zusammenfassung}
% 
\tableofcontents

\listoffigures

\listoftables

\begin{listofabbrv}{1000}
\item [USP] Universidade de S�o Paulo
% \item [CFS] Courtois-Finiasz-Sendrier
\end{listofabbrv}

\begin{listofsymbols}{1000}
\item [$\Delta(h)$] Assinatura di�dica
\end{listofsymbols}

%/Capitulo 1.
\chapter{Introducao} % (fold)
\label{cha:introducao}

A evolu��o da sociedade vem promovendo gradativamente diversas mudan�as entre elas est�o o aumento excessivo da popula��o, junto a falta de infraestrutura (ruas esburacadas, falta de sinaliza��o), m� localiza��o de moradia e trabalho, onde resultam em causas que afetam trafego urbano diariamente e provocam falta de organiza��o nas metr�poles, a soma desses fatores implicam na mobilidade urbana. 
No Brasil, a mobilidade urbana � uma �rea com infraestrutura prec�ria, onde o cidad�o, na grande maioria tem que se mover de um ponto da cidade para outro, necessitando de qualidade e rapidez, todavia essa  infraestrutura n�o segue o crescimento conforme a demanda, gerando transtornos como tr�nsito excessivo, congestionamento e acidentes. Como de costume parte da popula��o usufrui do transporte coletivo, devido as condi��es financeiras e facilidade de acesso, h� de se ressaltar tamb�m que esse tipo de transporte, acarreta em diversos problemas como superlota��o, falta de qualidade e seguran�a.
O avan�o computacional vem contribuindo para minimizar alguns destes problemas citados acima. Com a tecnologia � poss�vel realizar um melhor planejamento urbano, realizando buscas atrav�s de mapas ou tecnologia GPS, pode-se obter localiza��o exata de vias, pontos de congestionamentos ou acidentes, localiza��o de transportes p�blicos, para melhor escolher seu trajeto e melhorar consideravelmente o modo em que a informa��o circula.

% chapter introducao (end){Introdu��o}



%\chapter{Introdu��o}\label{chp:intro}
%
%\section{Apresenta��o}\label{sec:presentation}
%
%\section{Objetivos}\label{sec:goals}
%
%\section{Contribui��es originais}\label{sec:contrib}
%
%\begin{example}
%Isto � um exemplo.
%\qed
%\end{example}
%
%\section{Organiza��o}
%
%\chapter{Outro cap�tulo}\label{chp:outrocap}
%
%Este cap�tulo desenvolve a teoria de~\cite{agashe-lauter-venkatesan,al-riyami-paterson,ansi-x9.62,balasubramanian-koblitz,blake-seroussi-smart,bleichenbacher,weimerskirch}.
%
%\[
%E = mc^2.
%\]
%
%\chapter{Conclus�es}\label{chp:conclusion}
%
\bibliography{prat}
%
%\appendix
%
%\chapter{Demonstra��o do Lema da Bifurca��o}\label{app:apendiceA}

\end{document}
